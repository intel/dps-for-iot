\hypertarget{building-and-running_prerequisites}{}\section{Prerequisites}\label{building-and-running_prerequisites}
\hypertarget{building-and-running_prerequisites-linux}{}\subsection{Linux}\label{building-and-running_prerequisites-linux}

\begin{DoxyItemize}
\item gcc or clang
\item \href{http://scons.org/pages/download.html}{\tt S\+Cons}
\item libuv is used by node.\+js so packages are available for many distributions but note that D\+PS requires libuv 1.\+7 or later so it may be necessary to build libuv from source. \href{https://github.com/libuv}{\tt libuv source code on Git\+Hub.}
\item \href{http://www.swig.org/download.html}{\tt S\+W\+IG}
\end{DoxyItemize}\hypertarget{building-and-running_prerequisites-windows}{}\subsection{Windows}\label{building-and-running_prerequisites-windows}

\begin{DoxyItemize}
\item \href{https://www.visualstudio.com/downloads/}{\tt Visual Studio}

Note\+: In Visual Studio 2015, Visual C++ is not installed by default. When installing, be sure to choose {\bfseries Custom} installation and then choose the C++ components you require. Or, if Visual Studio is already installed, choose {\bfseries File $\vert$ New $\vert$ Project $\vert$ C++} and you will be prompted to install the necessary components.
\item \href{https://www.python.org/downloads/windows/}{\tt Latest Python 2.\+7 Release}
\item \href{http://scons.org/pages/download.html}{\tt S\+Cons}

Note\+: The S\+Cons installer will not detect the 64-\/bit installation of Python. Instead, download the zip file and follow the installation instructions in S\+Cons R\+E\+A\+D\+M\+E.\+txt.
\item \href{http://dist.libuv.org/dist/}{\tt libuv}
\item \href{http://www.swig.org/download.html}{\tt S\+W\+IG}
\end{DoxyItemize}\hypertarget{building-and-running_prerequisites-yocto}{}\subsection{Yocto}\label{building-and-running_prerequisites-yocto}
Yocto Project through the Open\+Embedded build system provides an open source development environment targeting the A\+RM, M\+I\+PS, Power\+PC and x86 architectures for a variety of platforms including x86-\/64 and emulated ones.


\begin{DoxyItemize}
\item \href{https://git.yoctoproject.org/}{\tt Yocto git}
\item \href{http://www.yoctoproject.org/docs/1.8/yocto-project-qs/yocto-project-qs.html}{\tt Yocto Project Quick Start}
\item \href{https://layers.openembedded.org/layerindex/recipe/32082/}{\tt Yocto libuv}
\end{DoxyItemize}\hypertarget{building-and-running_prerequisites-documentation}{}\subsection{Documentation}\label{building-and-running_prerequisites-documentation}
The C A\+PI documentation is generated using Doxygen. The Python (pydoc) and Java\+Script A\+PI (J\+S\+Doc) documentation is generated from the Doxygen output.

Doxygen can be downloaded from here\+: \href{http://www.stack.nl/~dimitri/doxygen/download.html}{\tt Doxygen}

Building the documentation requires the scons \href{https://bitbucket.org/scons/scons/wiki/DoxygenBuilder}{\tt Doxygen\+Builder} tool. This \href{https://bitbucket.org/scons/scons/wiki/ToolsIndex}{\tt page} has instructions on how to install the builder.\hypertarget{building-and-running_building}{}\section{Building}\label{building-and-running_building}
\hypertarget{building-and-running_building-linux-and-windows}{}\subsection{Linux and Windows}\label{building-and-running_building-linux-and-windows}
To build the D\+PS libraries, examples, bindings, and documentation run {\ttfamily scons}.

\begin{DoxyVerb}$ scons [variant=debug|release] [transport=udp|tcp|dtls] [bindings=all|none]
\end{DoxyVerb}


To build with a different compiler use the {\ttfamily CC} and {\ttfamily C\+XX} build options.

\begin{DoxyVerb}$ scons CC=clang CXX=clang++
\end{DoxyVerb}


To see the complete list of build options run {\ttfamily scons --help}. The default build configuration is {\ttfamily variant=release transport=udp bindings=all}.

\begin{DoxyNote}{Note}
A limitation of the current implementation is that the transport must be configured at compile time.
\end{DoxyNote}
The scons script pulls down source code from two external projects (mbedtls, and safestringlib) into the {\ttfamily ./ext} directory. If necessary these projects can be populated manually\+:

\begin{DoxyVerb}$ git clone https://github.com/ARMmbed/mbedtls ext/mbedtls
$ git clone https://github.com/01org/safestringlib.git ext/safestring
\end{DoxyVerb}


\begin{DoxyNote}{Note}
The ext projects are populated the first time D\+PS is built. To update these projects you need to manually do a {\ttfamily git pull} or delete the project directory and rerun scons.
\end{DoxyNote}
\hypertarget{building-and-running_building-yocto}{}\subsection{Yocto}\label{building-and-running_building-yocto}
Clone the poky repository and configure the Yocto environment. Refer to \href{http://www.yoctoproject.org/docs/1.8/yocto-project-qs/yocto-project-qs.html}{\tt Yocto Project Quick Start} for more information.

Clone the libuv Yocto project and yocto/recipes-\/connectivity/dps to the Yocto Project directory. Modify the value of S\+R\+C\+R\+E\+V\+\_\+dps in dps\+\_\+git.\+bb to the last commit of dps.

The Yocto Project directory needs to be included in B\+B\+L\+A\+Y\+E\+RS of conf/bblayers.\+conf. Refer to \href{https://wiki.yoctoproject.org/wiki/How_do_I}{\tt Yocto Wiki} for more information.

From the root directory of the Yocto Project, initialize the Yocto environment, provide a meaningful build directory name and build Yocto D\+PS.

\begin{DoxyVerb}$ source oe-init-build-env mybuilds
$ bitbake dps
\end{DoxyVerb}
\hypertarget{building-and-running_running}{}\section{Running}\label{building-and-running_running}
\hypertarget{building-and-running_running-examples}{}\subsection{Examples}\label{building-and-running_running-examples}
There are C, Python, and JS (node.\+js) examples.

The C examples are found in {\ttfamily ./examples}, the Python examples are in {\ttfamily ./py\+\_\+scripts} and the JS examples are in {\ttfamily ./js\+\_\+scripts}.

The C examples are installed in {\ttfamily ./build/dist/bin}. There are some some test scripts in {\ttfamily }./test\+\_\+scripts that run some more complex scenarios using the example programs. The test script {\ttfamily tree1} builds a small mesh and shows how publications sent to any node in the mesh get forwarded to the matching subscribers. The script {\ttfamily reg1} uses the {\ttfamily registry}, {\ttfamily reg\+\_\+pubs}, and {\ttfamily reg\+\_\+subs} examples programs to build a dynamic mesh using the experimental discovery service. 